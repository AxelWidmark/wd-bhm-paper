% mnras_template.tex 
%
% LaTeX template for creating an MNRAS paper
%
% v3.0 released 14 May 2015
% (version numbers match those of mnras.cls)
%
% Copyright (C) Royal Astronomical Society 2015
% Authors:
% Keith T. Smith (Royal Astronomical Society)

% Change log
%
% v3.0 May 2015
%    Renamed to match the new package name
%    Version number matches mnras.cls
%    A few minor tweaks to wording
% v1.0 September 2013
%    Beta testing only - never publicly released
%    First version: a simple (ish) template for creating an MNRAS paper

%%%%%%%%%%%%%%%%%%%%%%%%%%%%%%%%%%%%%%%%%%%%%%%%%%
% Basic setup. Most papers should leave these options alone.
\documentclass[fleqn,usenatbib]{mnras}

% MNRAS is set in Times font. If you don't have this installed (most LaTeX
% installations will be fine) or prefer the old Computer Modern fonts, comment
% out the following line
\usepackage{newtxtext,newtxmath}
% Depending on your LaTeX fonts installation, you might get better results with one of these:
%\usepackage{mathptmx}
%\usepackage{txfonts}

% Use vector fonts, so it zooms properly in on-screen viewing software
% Don't change these lines unless you know what you are doing
\usepackage[T1]{fontenc}
\usepackage{ae,aecompl}


%%%%% AUTHORS - PLACE YOUR OWN PACKAGES HERE %%%%%
% Only include extra packages if you really need them. Common packages are:
\usepackage{graphicx}	% Including figure files
\usepackage{amsmath}	% Advanced maths commands
\usepackage{amssymb}	% Extra maths symbols
\let\plate\undefined
\usepackage{tikz}
\usetikzlibrary{bayesnet}

%%%%%%%%%%%%%%%%%%%%%%%%%%%%%%%%%%%%%%%%%%%%%%%%%%

%%%%% AUTHORS - PLACE YOUR OWN COMMANDS HERE %%%%%

\newcommand{\aw}[1]{\textcolor{red}{[AW: #1]}}
\newcommand{\hvp}[1]{\textcolor{green}{[HVP: #1]}}
\newcommand{\dm}[1]{\textcolor{magenta}{[DM: #1]}}


\newcommand{\popp}{\boldsymbol{\Psi}}
\newcommand{\objp}{\boldsymbol{\psi}}
\newcommand{\data}{\mathbf{d}}
\newcommand{\Teff}{T}
\newcommand{\logg}{g}
\newcommand{\pr}{\text{Pr}}
\newcommand{\de}{\text{d}}
\newcommand{\kpc}{\text{kpc}}
\newcommand{\K}{\text{K}}


% Please keep new commands to a minimum, and use \newcommand not \def to avoid
% overwriting existing commands. Example:
%\newcommand{\pcm}{\,cm$^{-2}$}	% per cm-squared

%%%%%%%%%%%%%%%%%%%%%%%%%%%%%%%%%%%%%%%%%%%%%%%%%%

%%%%%%%%%%%%%%%%%%% TITLE PAGE %%%%%%%%%%%%%%%%%%%

% Title of the paper, and the short title which is used in the headers.
% Keep the title short and informative.
\title[Inferring properties of the white dwarf population]{Inferring properties of the local white dwarf population in astrometric and photometric surveys}

% The list of authors, and the short list which is used in the headers.
% If you need two or more lines of authors, add an extra line using \newauthor
\author[A. Widmark et al.]{
Axel Widmark,$^1$\thanks{E-mail: axel.widmark@fysik.su.se} 
Daniel Mortlock,$^{2,3}$
Hiranya V. Peiris$^{1,4}$
\\
% List of institutions
$^1$The Oskar Klein Centre for Cosmoparticle Physics, Department of
Physics, Stockholm University, AlbaNova, 10691 Stockholm, Sweden\\
$^2$Astrophysics Group, Imperial College London, Blackett Laboratory, Prince Consort Road, London SW7 2AZ, UK\\
$^3$Department of Astronomy, Stockholm University, AlbaNova, SE-10691 Stockholm, Sweden\\
$^4$Department of Physics and Astronomy, University College London, Gower Street, London, WC1E 6BT, UK\\
}

% These dates will be filled out by the publisher
\date{Accepted XXX. Received YYY; in original form ZZZ}

% Enter the current year, for the copyright statements etc.
\pubyear{2018}

% Don't change these lines
\begin{document}
\label{firstpage}
\pagerange{\pageref{firstpage}--\pageref{lastpage}}
\maketitle

% Abstract of the paper
\begin{abstract}
The \emph{Gaia} mission will provide precise astrometry for an unprecedented number of white dwarfs. With such a large data set, it is possible to infer properties of the white dwarf population using only astrometric and photometric information. We demonstrate a framework to accomplish this using a mock data set with SDSS \emph{ugriz} photometry and \emph{Gaia} astrometric information.
Our technique utilises a Bayesian hierarchical model for inferring properties of a white dwarf population while also taking into account all observational errors of individual stars. We demonstrate that astrometry significantly improves the statistical inference, leading to more robust conclusions which are less sensitive to systematic errors.
\aw{Should I delete the following sentence then?} This is a powerful method for constraining stellar evolution, especially Type 1a supernovae progenitor scenarios, as well as the star formation and dynamical history of the Milky Way.
\end{abstract}

% Select between one and six entries from the list of approved keywords.
% Don't make up new ones.
\begin{keywords}
white dwarfs -- stars: statistics -- astrometry
\end{keywords}

%%%%%%%%%%%%%%%%%%%%%%%%%%%%%%%%%%%%%%%%%%%%%%%%%%

%%%%%%%%%%%%%%%%% BODY OF PAPER %%%%%%%%%%%%%%%%%%







\section{Introduction}

White dwarfs (WDs) are the remnants of all light-to-intermediate mass stars \citep{1996ApJ...460..489R}. The local WD population carries information about the Galaxy's star formation history and dynamical history and constrain models of stellar evolution \citep{1987ApJ...315L..77W,2016NewAR..72....1G,2018arXiv180505849E}. WDs are known to be Type 1a thermonuclear supernovae progenitors, although the explosion mechanism is poorly understood \citep{Livio:2018rue}.

The Sloan Digital Sky Survey (SDSS) catalogues a total number of $\sim 29,000$ spectroscopically confirmed WDs \citep{2013ApJS..204....5K,2015MNRAS.446.4078K}. A fundamental difficulty in studying WDs is that they have no fixed mass or size, such that their observational properties are largely degenerate with distance. This degeneracy can be broken with high quality spectrometry and accurate atmospheric models, although this can be sensitive to systematic errors. The \emph{Gaia} mission, which recently published its second data release (DR2), is expected to increase the number of known WDs by approximately an order of magnitude \citep{2014A&A...565A..11C,Jordan:2006jg}. \emph{Gaia} will also provide astrometric information for local neighborhood WDs, which previous astrometric missions were unable to do. \emph{Hipparcos} had a limiting apparent magnitude of $V \sim 12.4$ \citep{1997A&A...323L..49P}, while \emph{Gaia} will see objects as dim as $G \sim 21$.

With the enormous size of the \emph{Gaia} data set, there is great potential in an approach that focuses on width rather than depth, by taking less detailed information into account but including a very large number of objects. In this article we demonstrate how to infer properties of the WD population in the solar neighborhood, using SDSS \emph{ugriz} photometry and \emph{Gaia} astrometry. We generate a mock data sample of WDs from a population model of temperature, surface gravity and spatial number density distributions. All sample objects has photometry and parallax information with observational errors expected from SDSS and \emph{Gaia}, and sample construction selection effects are taken into account.

In a WD population model, it is physically meaningful to divide the total population into WD sub-populations. White dwarfs form a family of phenomenological types, where the main classification is between DA and DB, depending on if the envelope is hydrogen or helium dominated \citep{Tremblay:2007hq,2011ApJ...737...28B,2015A&A...583A..86K}. DA and DB stars can be identified with accurate photometry, as demonstrated by \cite{Mortlock:2008gf}. We also discuss how to expand the statistical model when considering other WD sub-populations, such as separate disk and halo WD populations \citep{1998ApJ...503..239I,2016MNRAS.463.2453D}. We also discuss the possibility of identifying binary WD systems \citep{2018MNRAS.476.2584M} and demonstrate how to do so using photometric and astrometric information alone.

This paper is organized as follows. \aw{What should I write instead of 'In Sec...'?} In Sec.~\ref{sec:model} we outline our model for the WD population. In Sec.~\ref{sec:data} we describe the observational data that we consider. In Sec.~\ref{sec:method} we present out method of statistical inference. In Sec.~\ref{sec:mock} we generate a mock data catalogue and use our statistical method to infer the population model parameters from the data. In Sec.~\ref{sec:subpopulations} we discuss possible extensions to the WD model, such as differentiating between disk and halo subpopulations, as well the possibility of identifying double-degenerate binary WD systems. Finally, in Sec.~\ref{sec:discussion} we discuss and conclude.





\section{Model}\label{sec:model}

In this section we describe a model for the Milky Way WD population. This population model includes the WD spatial distribution, as well as distributions in intrinsic WD properties, parametrized by effective temperature, surface gravity, and type.

We list the population parameters of our model, encapsulated in a vector $\popp$, in Table \ref{tab:parameters}, along with the object parameters and data. The population parameters are $\alpha$, which parametrizing the distribution of temperatures, $\bar{g}$, $\sigma_g$, $\gamma_g$ which parametrize the distribution of surface gravities, and $f_\text{DB}$ which is the fraction of DB class WDs.

The distribution of surface gravity and effective temperature are parametrized by
\begin{equation}\label{eq:T&g}
\begin{split}
	\pr(\logg | \popp) & \propto \Theta(\logg \in [7,9]) \times f_t(\logg|\bar{g},\sigma_g,\gamma_g),\\
    \pr(\Teff | \popp) & \propto \Theta(\Teff/\K \in [1500,120000]) \times \exp (-\alpha \Teff),
\end{split}
\end{equation}
where $\Theta$ is a step function, $f_t(\logg)$ is Student's $t$-distribution of width $\sigma_g$ and variance $\text{Var}(g) = \gamma_g^2 \sigma_g^2$, such that the quantity $\gamma_g$ parametrizes the heaviness of the distribution's tails.

The class of the object constitutes a third parameter of the intrinsic WD properties, called $t$. This is not continuous but a label, denoting for example if the WD is of DA or DB classification. The probabilities are written in terms of the fraction of DB stars, like
\begin{equation}\label{eq:DADB}
\begin{split}
	\pr(t=\text{DA} | \popp) & = (1-f_\text{DB}),\\
    \pr(t=\text{DB} | \popp) & = f_\text{DB}.
\end{split}
\end{equation}

Given the intrinsic properties of a WD, the absolute magnitude in different photometric bands can be calculated using a stellar model. In this paper, we use the Bergeron atmospheric model for WDs \citep{Bergeron:1995we,Finley:1997zz,Bergeron:2000ce,2001PASP..113..409F}.

Also included in our model is a WD number density function, based on a Galactic model by \cite{2008ApJ...673..864J}, expressed in terms of cylindrical coordinates $R$, $Z$ and $\Phi$,
\begin{equation}\label{eq:numberdensity}
\begin{split}
	n(\mathbf{x}) \propto
	\Bigg[ 
		& \exp\Bigg(-\frac{R}{L_\text{thin}}\Bigg)\exp\Bigg(-\frac{|Z|}{H_\text{thin}}\Bigg) \\
		& +f_\text{thick}\exp\Bigg(-\frac{R}{L_\text{thick}}\Bigg)\exp\Bigg(-\frac{|Z|}{H_\text{thick}}\Bigg) \\
		& +f_\text{halo}\Bigg( \frac{(R^2+Z^2/q_\text{halo}^2+R_\text{core}^2)^{1/2}}{L_\text{halo}}^{-\eta_\text{halo}} \Bigg),
	\Bigg]
\end{split}
\end{equation}
where $f_\text{thick}=0.06$, $f_\text{halo}=6\times10^{-5}$, $L_\text{thin}=L_\text{thick}=3.5~\kpc$, $L_\text{halo}=8.5~\kpc$, $R_\text{core}=1~\kpc$, $H_\text{thin}=0.26~\kpc$, $H_\text{thick}=1~\kpc$, $q_\text{halo}=0.64$ and $\eta_\text{halo} = 2$. Assuming a solar position of $R_\odot=8~\kpc$ (where the height of the Sun above the plane is neglected), the Galactic coordinates are given by heliocentric coordinates through
\begin{equation}
\begin{split}
	& R(d,l,b) = [d^2\cos^2b-2 R_\odot d \cos^2b\cos^2+R_\odot^2]^{1/2}, \\
	& Z(d,l,b) = d \sin b.
\end{split}
\end{equation}
The azimuthal angle can be neglected, as the Galaxy is assumed to be axisymmetric in this model.



\begin{table}
	\centering
	\caption{The population parameters of our model, and the object parameters and data of the respective stars.}
	\label{tab:parameters}
    \begin{tabular}{l l}
		\hline
		$\popp$  & Population parameters \\
		\hline
		$\alpha$ & slope of temperature distribution \\
		$\bar{g}$ & mean surface gravity $\logg$ \\
		$\sigma_g$ & width of $\logg$ distribution \\
		$\gamma_g$ & heaviness of the tails of the $\logg$ distribution \\
		$f_\text{DB}$ & the fraction of DB type WDs \\
        \hline
        $\objp_{i=1,...,N}$  & Object parameters \\
        \hline
        $\Teff$ & effective temperature \\
        $\logg$ & surface gravity \\
        $t$ & WD classification (DA or DB) \\
        $\mathbf{x}$ & spatial position  \\
        \hline
        $\data_{i=1,...,N}$ & Data \\
        \hline
        $\hat{m}_b$ & observed photometric magnitude \\
        $\sigma_b$ & magnitude uncertainty \\
        $\hat{\varpi}$ & observed parallax \\
        $\sigma_{\hat{\varpi}}$ & parallax uncertainty \\
        $l$ & observed Galactic longitude \\
        $b$ & observed Galactic latitude \\
		\hline
	\end{tabular}
\end{table}




\section{Data}\label{sec:data}

We now demonstrate how to infer properties of a WD population given photometric and astrometric data. We begin by discussing the general case, showing to construct a likelihood for the properties of a WD. We then discuss the specific case of combining SDSS photometry and \emph{Gaia} astrometry.

The data for a given WD consist of apparent magnitude in some number of photometric bands ($\hat{m}_b$), a parallax measurement ($\hat{\varpi}$), angular position ($l$ and $b$), including the error models of these observables. A hatted quantity ($\hat{\varpi}$) refers to an observed value of some measurement error, while a non-hatted quantity ($\varpi$) refers to an observable's true value. The angles $l$ and $b$ are written without hats, to signify that their measurement uncertainties are so small that they can be neglected. The data of a WD with index $i$, called $\data_i$, is listed in Table \ref{tab:parameters}.

The likelihood associated with a WD, denoted by an index $i$, is written
\begin{equation}\label{eq:likelihood}
	\pr(\data_i | \objp_i) = \mathcal{N}(\varpi(\objp_i)|\hat{\varpi},\sigma_{\hat{\varpi}})\prod_{b} \mathcal{N}(m_b(\objp_i)|\hat{m}_b,\sigma_b),
\end{equation}
where the colour bands are iterated over in the product, and $\mathcal{N}(x | \mu,\sigma)$ is the normal distribution of mean $\mu$ and standard deviation $\sigma$. The factor containing parallax information is dropped when no parallax information is present. The apparent magnitudes $m_b(\objp_i)$ are functions of the object parameters, coming from a stellar model.

In this work, we use SDSS photometry in \emph{ugriz} colour bands, and a Bergeron atmospheric stellar model, as discussed in Sec. \ref{sec:model}. In order to assign realistic uncertainties to the mock data that we generate, we use median uncertainties of the SDSS DR9 catalogue, in bins of observed apparent magnitude of width 0.5 mag. These median uncertainties are visible in Fig.~\ref{fig:magnitude_error}. We limit the minimum \emph{ugriz} magnitude uncertainty to 0.01 mag, in order to account for possible systematic uncertainties in the Bergeron atmospheric model.

\begin{figure}
	\includegraphics[width=\columnwidth]{median_app_errors.pdf}
    \caption{The median magnitude errors in SDSS DR9, as a function of observed apparent magnitude.}
    \label{fig:magnitude_error}
\end{figure}

We use parallax information with the precision of \emph{Gaia} DR2, \citep{2018arXiv180409366L}. DR2 has a limiting magnitude of $m_G \simeq 21$, with parallax uncertainties of about $0.04$ mas for sources with $m_G<15$, about $0.1$ mas for sources with $m_G=17$, and about $0.7$ mas at $m_G=20$. In order to account for this magnitude dependence, we interpolate these points as shown in Fig. \ref{fig:parallax_error}. The errors in the \emph{Gaia} photometric $G$-band are interpolated in the same manner as for the parallax, with errors of 0.3 mmag for $m_G = 13$, 2 mmag for $m_G = 17$, and 10 mmag for $m_G = 20$.

\begin{figure}
	\includegraphics[width=\columnwidth]{parallax_error.pdf}
    \caption{Parallax uncertainties as a function of \emph{Gaia} $G$-band apparent magnitude $m_G$. The dots correspond to magnitudes $m_G=\{15,17,20\}$. For $m_G\leq 15$ the parallax uncertainty is 0.04 mas and constant.}
    \label{fig:parallax_error}
\end{figure}





\section{Method}\label{sec:method}

By Bayes' Theorem, the full posterior on both population parameters and object parameters is written
\begin{equation}\label{eq:fullposterior}
	\pr(\popp,\objp | \data ) = \pr(\popp)
    \prod_i \frac{S(\data_i) \pr(\data_i|\objp_i) \pr(\objp_i | \popp)}{\bar{N}(S,\popp)},
\end{equation}
where $\pr(\popp)$ is a prior on the population parameters; $S(\data_i)$ is the probability of being selected given the data of that object; $\pr(\data_i|\objp_i)$ is the likelihood of the data of an object given its object parameters; $\pr(\objp_i | \popp)$ is the probability of object parameters given the population parameters; finally, $\bar{N}(S,\popp)$ is the normalization of $\pr(\objp_i | \popp)$, and depends on the selection function and the population parameters. When writing the data or object parameters without an index $i$, we refer to the complete set of objects, $\objp \equiv \{ \objp_1,\objp_2,...,\objp_N \}$.

A directed acyclical graph (DAG) of our statistical model is shown in Fig.~\ref{fig:DAG}.

\begin{figure}\label{fig:DAG}
  \centering
  \tikz{
    \node[latent] (popp) {$\boldsymbol{\Psi}$} ; %
    \node[latent, right=of popp, rectangle] (selection) {$S$} ; %
    \node[latent, below=of selection] (norm) {$\bar{N}$} ; %
    \node[latent, below=of norm, rectangle] (data) {$\boldsymbol{d}$} ; %
    \node[latent, left=of data] (objp) {$\boldsymbol{\psi}$} ; %
    \plate[inner sep=0.3cm, xshift=-0.0cm, yshift=0.16cm] {index_i} {(data) (objp)} {i}; %
    \edge[dotted] {popp} {norm} ; %
    \edge[dotted] {selection} {norm} ; %
    \edge {norm} {objp} ; %
    \edge {popp} {objp} ; %
    \edge {data} {objp} ; %
  }
  \caption{A directed acyclical graph (DAG) of our statistical model. Quantities inscribed in circles (squares) represent variables (constants); solid (dotted) arrows represent probabilistic (deterministic) dependence; a rectangle with rounded corners represents a set of objects, in this case the set of WD included in our sample; $\popp$ is the set of population parameters, $S$ is the selection function of our sample construction, $\bar{N}$ is the normalization to the WD distribution function, $\objp$ is the set of object parameters, $\boldsymbol{d}$ is the object data, and $i$ is the object index.}
\end{figure}







\subsection{Object parameters}\label{sec:objectparams}

Each WD has a set of object parameters encapsulated in $\objp_i$, as listed in Table \ref{tab:parameters}. It consists of the effective temperature $\Teff$, the surface gravity $\logg$, the type $t$ denoting DA or DB classification, and $\Delta$ which parametrizes the distance.

Conceptually, the most straightforward parametrization would be to have the distance $d$ as the fourth object parameter. However, this creates some sampling difficulties arising from the fact that $\logg$ and $d$ are highly degenerate, especially when there is no parallax information that constrains the distance. In this work, we sample the object parameter posteriors using a Metropolis-Hastings Monte-Carlo Markov Chain \citep{1953JChPh..21.1087M,2018ApJS..236...11H}, which is more efficient when the posterior closer to a multivariate Gaussian in shape. This can be accomplished by a coordinate transformation, as is illustrated in Fig.~\ref{fig:banana}. In the upper panel, the distance is parametrized in terms of $\Delta$, which is the relative change to the ideal distance given $\Teff$ and $\logg$. Let $\tilde{d}(\Teff,\logg)$ be the distance that maximizes the colour factor of the likelihood function,
\begin{equation}
	\tilde{d} = 
    h^{-1}\left( \frac{\sum_b \sigma_b^{-2} (\hat{m}_c-M_c(\Teff,\logg))}{\sum_b \sigma_b^{-2}} \right),
\end{equation}
where $h^{-1}$ is the inverse of $h(d)=5[\log_{10}(d/\text{pc})-1]$, the difference between apparent and absolute magnitude. The distance parametrized by the object parameters is then $d=\Delta\tilde{d}(\Teff,\logg)$. It is crucial to account for the Jacobian factor that arises with this parametrization, which is $\de \Teff\, \de \logg\, \de d = \de \Teff\, \de \logg\, \de \Delta\, \tilde{d}(\Teff,\logg)$.

\begin{figure}
	\includegraphics[width=\columnwidth]{banana.pdf}
    \caption{A posterior density of an object with true parameters $\Teff=14000$ K, $\logg=7.9$, $t=\text{DA}$, and $d=50$ pc, with photometric errors of $\sigma_c=0.01$ in all \emph{ugriz} colour bands. The population parameters are set to $\alpha=3.5$, $\bar{g}=7.9$, $\sigma_g=0.1$, $\gamma_g=1.2$, and $f_\text{DB}=0$. Because there is no parallax information, the constraint to the surface gravity is largely due to the prior set by the population model. The top panel shows the posterior density over $\logg$ and $\Delta$, while the bottom panel shows the same posterior in a $\logg$ and $d$ parametrization.}
    \label{fig:banana}
\end{figure}







\subsection{Object posterior}\label{sec:objectposterior}

The posterior on a specific object also includes the term $\pr(\objp_i | \popp)$, normalized by the quantity $\bar{N}(S,\popp)$. It is given by the population, parametrized by the population parameters $\popp$,
\begin{equation}
\begin{split}
	& \pr(\objp_i | \popp)\, \de \Teff\, \de \logg\, \de d  \\ & \propto
    d^2 n(l,b,\varpi)\, \pr(\Teff,g,t | \popp)\, \de \Teff\, \de \logg\, \de d \\
    & = \Delta^2\, \tilde{d}^3(\Teff,\logg,t) n(l,b,\varpi)\, \pr(\Teff,g,t | \popp)\, \de \Teff\, \de\logg\, \de \Delta,
\end{split}
\end{equation}
where $n$ is the number density of WDs as a function of spatial position. It is implicit in this expression that the true parallax $\varpi$ is a function of the object parameters $\objp_i$.

The normalization factor is given by an integral (or sum, in the case of a discrete variable) over the possible properties of a hypothetical WD drawn from the population model, multiplied by the probability of being selected, like
\begin{equation}\label{eq:normalization}
	\bar{N}(S,\popp) = \sum_{t} \int \de\Teff\, \de \logg\, \de^3\mathbf{x}\,
    \pr(\Teff,g,t | \popp)\, n(\mathbf{x})\, S(\Teff,\logg,t,\mathbf{x}).
\end{equation}
The probability of being selected, $S(\Teff,\logg,t,\mathbf{x})$, is the probability of being included in the sample, given by an object's intrinsic properties and the sample construction cuts on observables. This is formulated in  the next section.







\section{Mock data and inference}\label{sec:mock}

To test the algorithm, mock data is generated from a population model, with population parameters $\alpha=3.5$, $\bar{g}=7.9$, $\sigma_g=0.1$, $\gamma_g=1.2$, and $f_\text{DB}=0.1$, and the WD number density $n(\mathbf{x})$ as described in Sec.~\ref{sec:model}.

While the exact values of these parameters are of lesser importance, as the main focus is to demonstrate the method, the chosen numerical values are motivated as follows. To first order, WDs cool at a rate of $\dot{\Teff} = \Teff^{-3.5}$, according to \cite{1952MNRAS.112..583M}. For cooler WDs, numerical models detract from this cooling rate. The surface gravity distribution mean and dispersion ($\bar{g}$ and $\sigma_g$) are well motivated, for example by \cite{2006ApJS..167...40E}. The tails of the surface gravity distribution are observed to be heavier than those of a Gaussian distribution, motivating a value $\gamma_g>1$. The fraction of DB WDs is observed to vary with temperature, roughly in the range of 10--20 per cent \citep{2011ApJ...737...28B}.

We compare the case of having no astrometric information, versus having parallax measurements with the precision of \emph{Gaia} DR2.

\subsection{Sample cuts}\label{sec:sample_cuts}

We define a sample by making cuts on observable properties, in our case on mock data of a generated catalogue. Depending on the errors of these observables, these cuts correspond more or less well to cuts in the objects' intrinsic properties. We do not make a volume limited sample by making cuts on parallax -- we want to compare to the case when astrometry is not available, hence we need to construct the sample without such information. We make cuts in observed apparent magnitude and observed colour. The cuts in colour correspond to upper and lower limits to the temperature of WDs included in our sample. The limit in apparent magnitude sets a maximum distance for a WD, as a function of temperature, surface gravity and type.

There are several reasons for not allowing very high temperatures in our samples (although the exact limit is rather arbitrary). Very hot objects are rare, but because they are seen to much further distance they can still be numerous in a sample that is not volume limited. How many are seen depends on the properties of the population, but this is degenerate with the assumed spatial distribution and the distribution of dust. Furthermore, with high enough temperature the peak of an object's spectrum is of shorter wavelength than the $u$ band, in which case the inference on an objects temperature becomes very weak. When working with actual data, it is also necessary to identify what objects are WDs and what objects are not. With very hot, far away objects this is difficult, especially since the distance will be poorly constrained. These issues can be circumvented with good parallax measurements, enabling the construction of a volume limited sample.

We make the following cut in colour, demanding that
\begin{equation}
	\hat{\delta}_{ugr} \equiv -0.4925\hat{m}_u-0.5075\hat{m}_g+\hat{m}_r \in (-0.6195,0.4369).
\end{equation}
Without measurement errors, this cut corresponds to limiting the temperature of a DA type WD to $\Teff \in (7000,30000)$ K; for a DB WD the upper limit in temperature is less restrictive, as can be seen in Fig.\ref{fig:colors_cut}.

\begin{figure}
	\includegraphics[width=\columnwidth]{colors_cut.pdf}
    \caption{Colours of a DA (blue grid) and DB (red grid) WD, in lines of constant $\Teff$ or $\logg$. The surface gravity takes values $\logg = \{7,7.5,8,8.5,9\}$. The black dashed lines corresponds to the colour cuts in $\hat{\delta}_{ugr}$, where only objects that fall in the region between these lines are included.}
    \label{fig:colors_cut}
\end{figure}

We also make a cut in brightness, by demanding that the \emph{Gaia} $G$ band apparent magnitude fulfills that
\begin{equation}
	\hat{m}_G < 20.
\end{equation}
In principle, this criteria could equally well have been formulated in terms of some combination of the $ugriz$ apparent magnitudes.

Given these cuts to observables, the selection function is written
\begin{equation}\label{eq:selection}
\begin{split}
	S(\Teff,\logg,t,\mathbf{x}) = 
    	      \int \de \hat{m}_G \Theta \big( 20-\hat{m}_G \big)\mathcal{N}\big( \hat{m}_G | m_G,\sigma_G \big) \\
    \times \int \de \hat{\delta}_{ugr} \Theta \big( \hat{\delta}_{ugr} \in (-0.6195,0.4369) \big) \mathcal{N}\big( \hat{\delta}_{ugr} | \delta_{ugr},\sigma_{\delta}\big),
\end{split}
\end{equation}
where $m_G$ and $\delta_{ugr}$ are true observables with an implicit dependence on the object parameters. The error on $\hat{\delta}_{ugr}$ is given by
\begin{equation}
	\sigma_\delta = \sqrt{ (0.4925 \sigma_u)^2 + (0.5075 \sigma_g)^2 + \sigma_r^2 },
\end{equation}
assuming that the different colour band errors are uncorrelated.


\subsection{Generating a mock data catalogue}

The mock catalogue is generated by rejection sampling. An object is drawn randomly from the true population model: the object parameters $\Teff$, $\logg$ and $t$ are distributed as described in Eq.~\eqref{eq:T&g} and Eq.~\eqref{eq:DADB} and can be randomized analytically; the position is distributed according to $n(\mathbf{x})$ and is randomized by rejection sampling, knowing that there is a maximal distance a WD can have to be included in our sample (observational errors included). A randomly drawn object is then assigned observable quantities, with errors as described in Sec.~\ref{sec:data}. If the object observables fulfill the selection cuts, the object is included in the sample; if not, it is rejected.

In this article, we construct a sample with 10,000 WDs. The distribution of true object parameter values is visible in Fig. \ref{fig:10000WDs}, where selection effects are manifest. The maximum distance is clearly seen as a function of temperature, where hotter objects are seen further away. Due to the colour cut, the high temperature tail is more pronounced for the DB sub-population. It is also clear that low mass WDs are more likely to be included as they are more luminous, giving rise to some asymmetry in the distribution of surface gravity. The hottest object in this sample has an effective temperature of $\Teff=39,060$ K and is of DB class. The most distant object is located at 1.56 kpc, is of DA classification and has an effective temperature $\Teff=37,287$ K surface gravity $\logg=7.72$.

\begin{figure*}
	\includegraphics[width=.9\textwidth]{10000WDs.pdf}
    \caption{The distributions of true intrinsic properties of our mock data sample, represented as 1D and 2D histograms. The axis are shared between the figures, except for the vertical axis of the histograms. In the 2D histograms, there is no differentiation between DA and DB WDs. In the 1D histograms, the distribution of DA WDs are shown in blue, and DB WDs are shown in red.}
    \label{fig:10000WDs}
\end{figure*}



\subsection{Results}

We infer the population parameters of a Bayesian hierarchical model, as described in Sec.~\ref{sec:model}, for our mock data sample, using a Monte-Carlo Markov Chain (MCMC) to trace the posterior distribution. We use a special purpose sampling algorithm called Metropolis-within-Gibbs, in which the population parameters and the object parameters are varied separately, producing a computationally efficient marginalization of the object parameters \citep{BayesianDataAnalysis}. All objects are modelled as independently drawn from the population model, so object parameters ($\objp_i$) are varied one object at a time. The WD classification type ($t$), which is a discrete variable, alternates at random for each attempted step of the object parameter Metropolis MCMC. For all other variables, which are continuous, the step is drawn at random from a multivariate Gaussian distribution. The covariance matrices of these multivariate Gaussians, for the population parameters and each of the 10,000 objects, are tuned in a thorough burn-in phase. For each iteration of the algorithm, the algorithm attempts 40 steps in population parameter space, and 40 steps in the parameter space of each separate object.

The inferred posterior distributions are visible in Fig. \ref{fig:chain} and \ref{fig:chain_parallax}, where the former has no parallax information. In each of these chains, the Metropolis-within-Gibbs MCMC has run for 10,000 iterations.

\begin{figure*}
	\includegraphics[width=1.\textwidth]{toy_chain.pdf}
    \caption{Posterior density of the population parameters, for a mock data set with no astrometric information.}
    \label{fig:chain}
\end{figure*}

\begin{figure*}
	\includegraphics[width=1.\textwidth]{toy_chain_include-parallax.pdf}
    \caption{Posterior density of the population parameters, for a mock data set with parallax information.}
    \label{fig:chain_parallax}
\end{figure*}

In both cases, with and without parallax information, the correct population parameter values are inferred to within statistical error. The inference on quantities $\alpha$ (parametrizing the distribution of effective temperature) and $f_\text{DB}$ (the fraction of DB WDs) is not much affected by also including parallax information. For the remaining parameters, $\bar{g}$, $\sigma_g$, and $\gamma_g$ (parametrizing the distribution of surface gravity), the inference is strongly affected and improved when including parallax information. The uncertainty is roughly halved for all three quantities. There is a clear degeneracy between $\sigma_g$ and $\gamma_g$, following the relation where the total variance on the surface gravity, $\text{Var}(\logg) = \sigma_g^2\gamma_g^2$, is preserved.







\section{White dwarf subpopulations}\label{sec:subpopulations}

The population of WDs in the Milky Way is richer and more complicated than the model described in Sec.~\ref{sec:model}. While we do consider WD sub-populations in the sense of accounting for the difference between DA and DB class WDs, there are other meaningful ways to compartmentalise the population model. While we assume that the distribution of temperatures and surface gravities is the same between DA and DB WDs, derived from the same population parameters, it could be meaningful to have separate sets of population parameters for the two classes.

In the same vein, one would expect the disk and halo WD population to have different properties. Furthermore, there is expected to be a sub-population of binary WD systems. These two different aspects of sub-populations are discussed below.



\subsection{Disk and halo populations}

It would be interesting to see qualitative differences between disk and halo WDs. For example, the kinematically warmer halo population has older stars and would therefore be colder. The population of very old WDs is scientifically interesting, as it holds information about the distant star formation and dynamical history of the Milky Way.

In this population model, each sub-population would have its own set of population parameters: $\popp = \{ \popp_\text{disk},\popp_\text{halo} \}$. They would each have their respective spatial number density distributions: $n_\text{disk}(\mathbf{x})$ and $n_\text{halo}(\mathbf{x})$. It would be necessary to have a population parameter that describes the relative number density fraction between the two sub-populations at some reference point (for example the Sun's position). The posterior, analogous to Eq.~\eqref{eq:fullposterior} would read
\begin{equation}\label{eq:posterior_disk_halo}
\begin{split}
	\pr(\popp,\objp | \data ) = \\ \pr(\popp)
	\prod_i \frac{S(\data_i) \pr(\data_i | \objp_i) \Big[ \pr(\objp_i | \popp_\text{disk})+\pr(\objp_i | \popp_\text{halo}) \Big]}{\bar{N}(S,\popp_\text{disk},\popp_\text{halo})}.
\end{split}
\end{equation}
The total number count is simply the sum over the two sub-populations, like
\begin{equation}
	\bar{N}(S,\popp_\text{disk},\popp_\text{halo})=\bar{N}_\text{disk}(S,\popp_\text{disk})+\bar{N}_\text{halo}(S,\popp_\text{halo}),
\end{equation}
where the relative number density fraction would be accounted for (included in $\popp_\text{halo}$, for example).






\subsection{Binary population}

Binary WD systems can be identified using only photometry and astrometry, in a similar way as what was used in \cite{2018ApJ...857..114W}. For a binary system, the likelihood is the same as is written in Eq.~\eqref{eq:likelihood}, the difference being that the \emph{ugriz} apparent luminosities of the two component stars are added together, according to
\begin{equation}
	m_\text{sum} = - \frac{5}{2}\log_{10}\left( 10^{-\frac{2}{5}m_{A}}+10^{-\frac{2}{5}m_{B}}  \right),
\end{equation}
where $m_A$ and $m_B$ are the apparent magnitudes of the two component stars, in the relevant photometric band.

The posterior density of a binary system will be written in terms of 7 parameters instead of 4, as we have temperature, surface gravity, and type of the two component stars, and the distance of the system.

We construct a population of mock binaries by random pairing of the singles population and the same selection criteria, although we also add a constraint in terms of cooling time of the two component stars. In addition to the effective temperature and surface gravity distributions of the two component stars, as described by Eq.~\eqref{eq:T&g}, the probability of pairing also has a factor
\begin{equation}\label{eq:time_difference}
	\exp\left(
	-\frac{[t_\text{cool}(\Teff_A,\logg_A,t_A)-t_\text{cool}(\Teff_B,\logg_B,t_B)]^2}{2\times ( 500~\text{Myrs})^2}
	\right),
\end{equation}
where $t_\text{cool}(\Teff_A,\logg_A,t_A)$ is the cooling time of the $A$ component WD (and equivalently for component $B$), as given by the Bergeron atmospheric model. The chosen time difference of $\sim 500$ million years prohibits the pairing of extremely cool and faint WDs with hotter ones. This is a reasonable assumption, as binary stars are typically born in the same system and with similar properties. Without this constraint, one component star would almost always be extremely faint, making binary identification almost impossible. For reference, the cooling time of a WD with $\Teff=10,000~\K$ and $\logg=7.9$ is roughly 500 Myr.

Figure \ref{fig:ROC_binaries} shows a receiver operator characteristic (ROC) curve for identification of binaries. The binaries are inferred with knowledge of the underlying population model, in the sense that the population parameters are known. The Bayesian evidence for being a single WD and being a binary WD is computed by tracing the posterior over the object parameters with an MCMC, given by a first order multivariate Gaussian approximation, computed by the covariance matrix and maximum posterior value of the MCMC points. The integral is computed for all possible DA and DB combinations separately, with a $\Teff_A>\Teff_B$ multiplicity constraint for binary WDs, circumventing issues of multimodal posterior densities. This is done for 1000 mock data single WDs and 1000 mock data binary WD systems. Computing the integral with a more sophisticated algorithm could give a marginally stronger identification power. It is clear from Fig.~\ref{fig:ROC_binaries} that some binaries can be very strongly identified, even with a very low contamination rate.
\begin{figure}
	\includegraphics[width=\columnwidth]{ROC_binaries.pdf}
    \caption{A receiver operator characteristic (ROC) curve of binary WD identification, showing the rate of falsely identified versus correctly identified binary WD systems in solid blue. An object is labelled binary if the Bayes factor is above some value, where this value varies along the graph. The black dashed line shows the linear relationship.}
    \label{fig:ROC_binaries}
\end{figure}

This identification is made on mock data when the underlying population model is known. Working with actual data comes with many complications, not least the fact that the population model is unknown and inferred. Even so, this test shows that it should be possible to identify a WD binary population. An important aspect that is not accounted for here is that some WD binaries are not drawn from the same distribution of masses as the population of single WDs. A tight binary system goes through phases of mass transfer and shared envelopes; thus there will be binaries with component WD with low mass and surface gravity. Such a binary system would actually be even easier to detect using this method, as they would be brighter due to multiplicity, and brighter still from being low mass and larger in size.







\section{Discussion}\label{sec:discussion}

In this article, we demonstrate how to infer properties of the local WD population using only astrometric and photometric information, in the framework of a Bayesian hierarchical model. We generate a mock data set with SDSS \emph{ugriz} photometry and \emph{Gaia} parallax information, and are able to retrieve the population parameters used to generate that sample.

In our mock data set, we have limited ourselves to a total number of 10,000 WDs. The catalogue of WD in a \emph{Gaia} and SDSS cross-match is expected to be around an order of magnitude larger, enabling us to fit a more complicated model. The model could be extended with more complicated distributions of effective temperature and surface gravity, and by including sub-populations as discussed in Sec.~\ref{sec:subpopulations}.

When working with actual data, there are complications that are not included here but would be straight forward to implement in this framework. In terms of dust, most WD seen by \emph{Gaia} and SDSS are very close and almost unaffected. However, hotter and more luminous WDs are seen to further distance and subject to dust reddening and extinction. With a good dust map, selection effects and photometric reddening for such objects can be accounted for. Also not included in this article are incompleteness effects, which are severe for WDs in \emph{Gaia} DR2. This will get significantly better with future data releases, but will still be crucial to account for. As formulated in Sec.~\ref{sec:sample_cuts}, the selection function of this article only contains selection effects coming from the sample cuts; any incompleteness effects can be easily added as a factor to the selection function, dependent on data and/or intrinsic WD properties. The inference on both object and population parameters can be biased if the stellar models are inaccurate. In such a case, a similar framework can be used to infer data-driven corrections to these models.

\emph{Gaia} parallax measurements provides very robust identification of WDs, enables the construction of volume limited samples, and breaks the degeneracy between distance and size. It is possible to differentiate sub-populations of WDs using this method, perhaps even inferring a population of binary WD systems. The uncertainty in inferring the properties of any single object is typically large, and it is absolutely essential to use a statistical model which fully and correctly accounts for such uncertainties. In this article, we have demonstrated how to do so in a framework of a Bayesian hierarchical model. In the near future, we plan to use this method to infer properties of the WD population of the Milky Way, using data from \emph{Gaia}, SDSS and potentially other photometric surveys.



\section*{Acknowledgements}

We would like to thank Pierre Bergeron, for providing WD atmospheric models also with Gaia photometry.
%%%%%%%%%%%%%%%%%%%%%%%%%%%%%%%%%%%%%%%%%%%%%%%%%%

%%%%%%%%%%%%%%%%%%%% REFERENCES %%%%%%%%%%%%%%%%%%

% The best way to enter references is to use BibTeX:

\bibliographystyle{mnras}
\bibliography{refs} % if your bibtex file is called example.bib

%%%%%%%%%%%%%%%%%%%%%%%%%%%%%%%%%%%%%%%%%%%%%%%%%%

%%%%%%%%%%%%%%%%% APPENDICES %%%%%%%%%%%%%%%%%%%%%

%\appendix
%\section{Some extra material}


%%%%%%%%%%%%%%%%%%%%%%%%%%%%%%%%%%%%%%%%%%%%%%%%%%


% Don't change these lines
\bsp	% typesetting comment
\label{lastpage}
\end{document}

% End of mnras_template.tex